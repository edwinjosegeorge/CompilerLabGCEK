\clearpage
\chapter{Generation of YACC specification for a few syntactic categories}

\section{Aim}
To design and implement YACC specification for the following syntactic categories.
\begin{enumerate}
	\item Program to recognize a valid arithmetic expression that uses operator +, – , * and /.
	\item Program to recognize a valid variable which starts with a letter followed by any	number of letters or digits.
	\item Implementation of Calculator using LEX and YACC
	%\item Convert the BNF rules into YACC form and write code to generate abstract syntax tree
\end{enumerate}


\section{Recognize Arithmetic expression}
\subsection{Algorithm}
\begin{algorithm}[H]
	\caption{An algorithm to recognize numbers , operators and variables}
	\begin{algorithmic}[1]
		\State $isNum \gets False$
		\State $isVar \gets False$
		\State $buf \gets ""$
		\State $i \gets 0$
		\State $buf[i] \gets $ inputChar()
		\While{$buf[i]$ == whitespace}
		\State $buf[i] \gets $ inputChar()
		\EndWhile
		
		\If{ $buf[i]$ == operator }
		\State $return$ $buf[i]$,"operator"
		\ElsIf{ $buf[i]$ == '(' }
		\State $return$ $buf[i]$,"openBracket"
		\ElsIf{ $buf[i]$ == ')' }
		\State $return$ $buf[i]$,"closeBracket"
		\ElsIf{$buf[i]$ == alpha or underscore}
		\State $isVar \gets True$
		\ElsIf{$buf[i]$ == number}
		\State $isNum \gets True$
		\Else
		\State $return$ "ERROR"
		\EndIf
		\State $i \gets i+1$
		
		\While{$(buf[i]$ = inputChar()) $\neq$ EOF}
		\If{$isVar and buf[i]$ == (alphaNum or underscore)}
		\State $i \gets i + 1$
		\State continue
		\ElsIf{$isNum$ and $buf$ == number)}
		\State $i \gets i + 1$
		\State continue
		\Else
		\State unputChar($buf[i]$) \Comment{Put back the last char read}
		\State $buf[i]$ = $'\ '$
		\If{$isVar$}
		\State $return$ $buf$,"variable"
		\ElsIf{$isNum$}
		\State $return$ $buf$,"number"
		\Else
		\State $return$ "ERROR"
		\EndIf
		\EndIf
		\EndWhile
	\end{algorithmic}
\end{algorithm}

\begin{algorithm}[H]
	\caption{A grammar to recognize expression}
	\begin{algorithmic}[1]
		\State $FINAL\_EXPR \gets EXPR$ $NEWLINE$
		\State $EXPR \gets number$
		\State $EXPR \gets identifier$
		\State $EXPR \gets $openBracket $EXPR$ closeBracket
		\State $EXPR \gets EXPR$ operator $EXPR$
	\end{algorithmic}
\end{algorithm}

\subsection{Lex program}
\lstinputlisting[style=CStyle]{../EXP3/expression.l}
\subsection{Yacc Program}
\lstinputlisting[style=CStyle]{../EXP3/expression.y}
\subsection{Output}
Compiling program
\lstinputlisting[style=Terminal]{../EXP3/expression_run.sh}


\vspace{0.5cm}
Output
\lstinputlisting[style=plain]{../EXP3/expression_output.txt}


\section{Recognize letter followed by any number of letters or digits}
\subsection{Algorithm}
\begin{algorithm}[H]
	\caption{An algorithm to recognize variables}
	\begin{algorithmic}[1]
		\State $buf \gets $ inputChar()
		\If{$buf$ == letter}
		\State $return$ letter
		\ElsIf{$buf$ == digit}
		\State $return$ digit
		\ElsIf{$buf$ == newline}
		\State $return$ NEWLINE
		\Else
		\State $return$ error
		\EndIf
	\end{algorithmic}
\end{algorithm}

\begin{algorithm}[H]
	\caption{A grammar to recognize variables}
	\begin{algorithmic}[1]
		\State $FINAL\_VAR \gets letter\ VAR\ NEWLINE$
		\State $VAR \gets VAR\ letter$
		\State $VAR \gets VAR\ digit$
		\State $VAR \gets$ Epsilon
	\end{algorithmic}
\end{algorithm}

\subsection{Lex program}
\lstinputlisting[style=CStyle]{../EXP3/variable.l}
\subsection{Yacc Program}
\lstinputlisting[style=CStyle]{../EXP3/variable.y}
\subsection{Output}
Compiling program
\lstinputlisting[style=Terminal]{../EXP3/variable_run.sh}


\vspace{0.5cm}
Output
\lstinputlisting[style=plain]{../EXP3/variable_output.txt}

\section{Stimulate Calculator}
\subsection{Algorithm}
\begin{algorithm}[H]
	\caption{An algorithm to recognize numbers and operators}
	\begin{algorithmic}[1]
		\State $buf \gets ""$
		\State $i \gets 0$
		\State $buf[i] \gets $ inputChar()
		\While{$buf[i]$ == whitespace}
		\State $buf[i] \gets $ inputChar()
		\EndWhile
		
		\If{ $buf[i]$ == operator }
		\State $return$ $buf[i]$
		\ElsIf{ $buf[i]$ == NEWLINE }
		\State $return$ NEWLINE
		\ElsIf{ $buf[i] \neq$ digit }
		\State $return$ ERROR
		\EndIf
		
		\State $i \gets i+1$
		\While{$(buf[i]$ = inputChar()) == digit}
		\State $i \gets i + 1$
		\EndWhile
		
		\State unputChar($buf[i]$) \Comment{Put back the last char read}
		\State $buf[i]$ = $'\ '$
		\State $return$ convert\_to\_number($buf$)		
	\end{algorithmic}
\end{algorithm}

\begin{algorithm}[H]
	\caption{A grammar to compute expression}
	\begin{algorithmic}[1]
		\State $FINAL\_EXPR \gets EXPR$ $NEWLINE$ \Comment{print expression}
		\State $EXPR \gets number$ \Comment{RHS = LHS}
		\State $EXPR \gets $openBracket $EXPR$ closeBracket \Comment{RHS = valueOf(EXPR)}
		\State $EXPR \gets EXPR$ * $EXPR$ \Comment{RHS = EXPR1 * EXPR2}
		\State $EXPR \gets EXPR$ + $EXPR$ \Comment{RHS = EXPR1 + EXPR2}
		\State $EXPR \gets EXPR$ - $EXPR$ \Comment{RHS = EXPR1 - EXPR2}
		\State $EXPR \gets EXPR$ / $EXPR$ \Comment{if(EXPR2 != 0) then RHS = EXPR1 / EXPR2, else print "error"}
		
	\end{algorithmic}
\end{algorithm}

\subsection{Lex program}
\lstinputlisting[style=CStyle]{../EXP3/calculator.l}
\subsection{Yacc Program}
\lstinputlisting[style=CStyle]{../EXP3/calculator.y}
\subsection{Output}
Compiling program
\lstinputlisting[style=Terminal]{../EXP3/calculator_run.sh}


\vspace{0.5cm}
Output
\lstinputlisting[style=plain]{../EXP3/calculator_output.txt}

\section{Result}
Implemented and verified YACC specification for a few syntactic categories